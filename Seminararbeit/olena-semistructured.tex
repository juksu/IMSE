\section{Semistrukturierte Daten}
Semi-strukturierte Daten ist eine Form von strukturierten Daten, die nicht mit der formalen Struktur von Datenmodellen übereinstimmen, die mit relationalen Datenbanken oder anderen Formen von Datentabellen verknüpft sind, aber dennoch Tags oder andere Marker enthält, um semantische Elemente zu trennen und Hierarchien innerhalb der Daten zu erzwingen. Daher ist es auch als selbstbeschreibende Struktur bekannt.
Semi-strukturierte Daten sind Daten, die keine Rohdaten oder typisierte Daten in einem herkömmlichen Datenbanksystem sind. Es ist strukturierte Daten, aber es ist nicht in einem rationalen Modell, wie eine Tabelle oder ein Objekt-basierte Grafik organisiert. Viele im Web gefundene Daten können als semi-strukturiert beschrieben werden. Die Datenintegration nutzt vor allem halbstrukturierte Daten. \cite{buneman2010}
In halbstrukturierten Daten können die Entitäten, die zu derselben Klasse gehören, unterschiedliche Attribute haben, obwohl sie zusammen gruppiert sind. Die Reihenfolge der Attribute ist nicht wichtig.    
Seit dem Aufkommen des Internets gibt es immer mehr strukturierte Daten, bei denen Volltextdokumente und Datenbanken nicht mehr die einzigen Datenformen sind und unterschiedliche Anwendungen ein Medium für den Informationsaustausch benötigen. In objektorientierten Datenbanken findet man oft halbstrukturierte Daten. \cite{abiteboul2000}
Arten von Semi-strukturierten Daten 

\subsection{XML}

XML, andere Markup-Sprachen, E-Mail und EDI sind alle Formen von halbstrukturierten Daten. OEM (Object Exchange Model) wurde vor XML als Mittel zur Selbstbeschreibung einer Datenstruktur erstellt. XML wurde von Webdiensten, die mit SOAP-Prinzipien entwickelt wurden, popularisiert.
Einige Arten von Daten, die hier als "semi-strukturiert", vor allem XML, beschrieben werden, leiden unter dem Eindruck, dass sie nicht in der Lage sind, strukturelle Strenge auf der gleichen funktionalen Ebene wie Relational Tables und Rows zu erhalten. In der Tat hat die Sicht von XML als inhärent halb-strukturiert (bisher wurde es als "unstrukturiert" bezeichnet) seine Verwendung für eine erweiterte Palette von datenzentrischen Anwendungen behindert. Sogar Dokumente, die normalerweise als der Inbegriff der Semi-Struktur gedacht sind, können mit praktisch der gleichen Strenge wie Datenbank-Schema entworfen werden, durch das XML-Schema erzwungen und von kommerziellen und kundenspezifischen Softwareprogrammen verarbeitet werden, ohne ihre Benutzerfreundlichkeit zu reduzieren. \cite{boneman2007}
Angesichts dieser Tatsache könnte XML als "flexible Struktur" bezeichnet werden, die in der Lage ist, menschlich-zentrische Strömung und Hierarchie sowie eine sehr rigorose Elementstruktur und Datentypisierung zu ermöglichen.
Das Konzept von XML als "menschlich lesbar" kann jedoch nur so weit genommen werden. Einige Implementierungen / Dialekte von XML, wie die XML-Darstellung des Inhalts eines Microsoft Word-Dokuments, wie sie in Office 2007 und späteren Versionen implementiert sind, verwenden Dutzende oder sogar Hunderte von verschiedenen Arten von Tags, die eine bestimmte Problemdomäne widerspiegeln - im Fall von Word, Formatierung auf der Charakter- und Absatz- und Dokumentenebene, Definitionen von Stilen, Einbeziehung von Zitaten usw. - die in komplexer Weise ineinander verschachtelt sind. Das Verständnis eines solchen XML-Dokuments, ist unmöglich, ohne ein sehr tiefes vorheriges Verständnis der spezifischen XML-Implementierung, zusammen mit Hilfe von Software, die das verwendete XML-Schema zu verstehen. Dieser Text ist nicht mehr "menschlich verständlich" als ein Buch. Die Tags sind Symbole, die bedeutungslos für eine Person sind, die mit der Domäne nicht arbeitet. \cite{lore}


\subsection{JSON}

JSON oder JavaScript Object Notation ist ein offenes Standardformat, das menschlich lesbaren Text verwendet, um Datenobjekte aus Attributwertpaaren zu übertragen. Es wird hauptsächlich verwendet, um Daten zwischen einem Server und einer Webanwendung als Alternative zu XML zu übertragen. JSON wurde von Web-Services entwickelt, die unter Verwendung von REST-Prinzipien entwickelt wurden.
Es gibt eine neue Zucht von Datenbanken wie MongoDB und Couchbase, die Daten nativ im JSON-Format speichern und die Profis der halbstrukturierten Datenarchitektur nutzen. \cite{buneman2010}

\subsection{Vor- und Nachteile der Verwendung eines halbstrukturierten Datenformats}
\subsubsection{Vorteile}
Programmierer, die Objekte von ihrer Anwendung auf eine Datenbank aufhalten, müssen sich nicht um objekt-relationale Impedanzfehlanpassung kümmern, sondern können oft Objekte über eine leichte Bibliothek serialisiert werden.
Die Unterstützung für verschachtelte oder hierarchische Daten vereinfacht oft Datenmodelle, die komplexe Beziehungen zwischen Entitäten darstellen.
Die Unterstützung für Listen von Objekten vereinfacht Datenmodelle durch Vermeidung von unordentlichen Übersetzungen von Listen in ein relationales Datenmodell.
Es kann die Information einiger Datenquellen darstellen, die nicht durch Schema eingeschränkt werden können.
Es bietet ein flexibles Format für den Datenaustausch zwischen verschiedenen Arten von Datenbanken.
Es kann hilfreich sein, strukturierte Daten als semi-strukturierte (für Browsing-Zwecke) zu sehen.
Das Schema kann leicht geändert werden.
Das Datenübertragungsformat kann tragbar sein. \cite{boneman2007}
\subsection{Nachteile}
Das traditionelle relationale Datenmodell hat eine populäre und fertige Abfragesprache, SQL.
Anfällig für "Müll in, Müll raus"; Durch das Entfernen von Beschränkungen aus dem Datenmodell gibt es weniger Vorbedenken, die notwendig sind, um eine Datenanwendung zu betreiben.
Das halbstrukturierte Modell ist ein Datenbankmodell, bei dem es keine Trennung zwischen den Daten und dem Schema gibt und die Menge der verwendeten Strukturen vom Zweck abhängt.
Der primäre Kompromiss, der bei der Verwendung eines semi-strukturierten Datenbankmodells gemacht wird, ist, dass Abfragen nicht so effizient gemacht werden können wie in einer beschränkten Struktur, wie zum Beispiel im relationalen Modell. Typischerweise werden die Datensätze in einer halbstrukturierten Datenbank mit eindeutigen IDs gespeichert, auf die mit Zeigern auf ihren Speicherort verwiesen wird. Dies macht Navigations- oder Pfad-basierte Abfragen recht effizient, aber für die Suche über viele Datensätze (wie es in SQL typisch ist), ist es nicht so effizient, weil es um die Festplatte nach Zeigern suchen muss. \cite{bry2001}
Das Object Exchange Model (OEM) ist ein Standard, um semi-strukturierte Daten zu erstellen, ein anderer Format ist XML.
\subsection{Wie kann man halbstrukturierte Daten verwalten?}
In unserer sich schnell verändernden IT-Welt wird es immer wichtiger, sich über verschiedene Datenformen zu informieren und wie (oder wenn) Sie sie verwalten müssen.
Strukturierte Daten sind Daten, die in kleine, diskrete Einheiten aufgeteilt wurden. Jede Datenmenge betrifft eine Sache (um ein gutes angelsächsisches Fangwort zu benutzen), zum Beispiel den Nachnamen eines Kunden. Strukturierte Daten werden typischerweise in Tabellen gespeichert. Wenn wir mit unserem Beispiel fortfahren, würde eine Spalte von Daten die Nachnamen aller Kunden auflisten, und jede Zeile würde sich auf einen Kunden beziehen. Diese Tabellen werden in der Regel in einer relationalen Datenbank gespeichert. \cite{lore}
In sehr vielen Fällen finden wir, dass Daten in der realen Welt nicht ganz so gut strukturiert sind. Aber wir veranlassen die Datenbank-Struktur auf sie aus dem einfachen Grund, dass dies die Daten leicht abzurufen und abzufragen macht. In der Praxis funktioniert das gut für die Verwaltung der meisten Geschäftsdaten; Aktienkontrolle, Finanzen, Human Resources und andere Unternehmenssysteme reichen sich ganz leicht einer auferlegten Datenstruktur zu.
Das Problem ist, dass einige Daten nicht zu einer rigorosen Strukturierung zugänglich sind - und diese Daten werden immer häufiger. Eine Vielzahl von Daten, die für das Unternehmen relevant sind, wird in Dokumenten, Bildern und E-Mails sowie Tweets und anderen Social Media Daten auftauchen. Alle diese Daten können als halbstrukturierte Daten beschrieben werden. \cite{abiteboul2000}

Unsere Möglichkeiten zur Verwaltung von halbstrukturierten Daten sind
\begin{enumerate}
	\item Ignorieren
	\item Erzwinge es in strukturierte relationale Form
	\item einen anderen Speichermechanismus annehmen
\end{enumerate}

\paragraph{Ignorieren} So viele Daten werden erstellt und gesammelt in halbstrukturierten Formen, die die meisten Unternehmen nicht leisten können, die Ausgießung davon zu ignorieren. Dies ist nur dann möglich, wenn es keinen zwingenden Geschäftsvorteil gibt, solche Daten zu verfolgen und zu analysieren.
\paragraph{Bleiben Sie relational} Relationale Datenbanken wurden signifikant verändert, um das zu behandeln, was sich von den Datenbankherstellern als "komplexe Datentypen" auszeichnet. XML ist ein Beispiel: Es wird von vielen als eine hervorragende Möglichkeit angesehen, klassische, halbstrukturierte Daten zu halten. Die meisten gängigen Dokumentenformate sind oder können in XML gerendert werden, und fast alle relationalen Motoren haben nun einen XML-Datentyp, was bedeutet, dass Dokumente oft in einer relationalen Datenbank gespeichert werden können. Aber die zusätzliche Komplexität der Handhabung von semi-strukturierten Daten bedeutet, dass es unvermeidlich ein Kompromiss sein wird, und im Allgemeinen wird das mit langsamen Abrufzeiten gleichsetzen. 
\paragraph{Annahme eines anderen Ansatzes} Es gibt zunehmendes Interesse an der Annahme alternativer Datenmanagement- und Speichermechanismen. Stellen Sie sich vor, Sie speichern Patienten Röntgenbilder als Bilder. Wir speichern Daten, damit wir sie später abholen können und auch so können wir sie abfragen, aber eine Abfrage gegen ein Röntgenbild ist ein etwas bizarres Konzept, weil das Röntgenbild einfach eine Sammlung von Pixeln ist. Was in der Praxis oft passiert, ist, dass diese und andere semi-strukturierte Daten mit einigen angehängten Metadaten kommen und auch irgendeine Form der Analyse durchlaufen können, um weitere Metadaten zu erzeugen. (Kurz gesagt, Metadaten sind Daten über Daten). Im Falle einer E-Mail können die angehängten Metadaten Länge, Absender, Empfänger, Uhrzeit / Datum und so weiter enthalten. \cite{bry2001}
Und nun über Röntgenstrahlen nachdenken, die klassische halbstrukturierte Daten sind. Während man sich nicht ein rohes Röntgenbild abfragen würde, kann man seine Metadaten abfragen. Die angehängten Metadaten könnten Patienten-ID, Arzt-ID, umfangreiche Informationen darüber, wie und wann das Röntgenbild genommen wurde und so weiter. Automatische Analyse des Bildes könnte Metadaten wie Diagnose, Prognose sein. In dieser Lösung können die halbstrukturierten Daten einfach als Bilddateien im Dateisystem gespeichert werden und die strukturierten Metadaten werden in einer relationalen Datenbank gespeichert und mit dem Bild verknüpft. Eine Abfrage könnte dann alle Röntgenstrahlen für den Arzt ID herausziehen, die gebrochene Gliedmaßen beinhalten und die Bilder anzeigen. \cite{abiteboul2000}
Semi-strukturierte Daten bieten das Potenzial für Geschäftsvorteil für jedes Unternehmen, das sie gut behandelt und analysiert.

